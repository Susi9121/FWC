#Python libraries for math and graphics
import numpy as np
import matplotlib.pyplot as plt
from numpy import linalg as LA

import sys                                          #for path to external scripts
sys.path.insert(0, '/home/susi/CoordGeo')        #path to my scripts

#local imports
from line.funcs import *
from triangle.funcs import *
from conics.funcs import circ_gen

#if using termux
import subprocess
import shlex
#end if
#Given points
h=float(input("Enter value of h"))
a=float(input("Enter value of a"))
b=float(input("Enter value of b"))
k=float(input("Enter value of k"))
A = np.array(([h,0]))
B = np.array(([a,b]))
C=np.array(([0,k]))
D=a/h+b/k
print("the a/h+b/k =",D)

##Generating all lines
x_AB = line_gen(A,B)
x_BC=line_gen(B,C)


#Plotting all lines
plt.plot(x_AB[0,:],x_AB[1,:],label='$AB$')
plt.plot(x_BC[0,:],x_BC[1,:],LABEL='$BC$')

#Labeling the coordinates
tri_coords = np.vstack((A,B,C)).T
plt.scatter(tri_coords[0,:], tri_coords[1,:])
vert_labels = ['A','B','c']
for i, txt in enumerate(vert_labels):
    plt.annotate(txt, # this is the text
                 (tri_coords[0,i], tri_coords[1,i]), # this is the point to label
                 textcoords="offset points", # how to position the text
                 xytext=(0,10), # distance from text to points (x,y)
                 ha='center') # horizontal alignment can be left, right or center

plt.xlabel('$x$')
plt.ylabel('$y$')
plt.legend(loc='best')
plt.grid() # minor
plt.axis('equal')

#if using termux
#plt.savefig('/storage/emulated/0/github/cbse-papers/2020/math/10/solutions/figs/matrix-10-2.pdf')
#subprocess.run(shlex.split("termux-open '/storage/emulated/0/github/cbse-papers/2020/math/10/solutions/figs/matrix-10-2.pdf'")) 
#else
plt.show()

